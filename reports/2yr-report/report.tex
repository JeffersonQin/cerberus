%!TEX TS-program = xelatex
%!TEX encoding = UTF-8 Unicode

\documentclass[11pt,a4paper,twocolumn]{article}

\usepackage[british]{babel}
\bibliographystyle{alpha}

\usepackage{amsmath}
\usepackage{amsthm}
\usepackage{amssymb}

% \usepackage{xltxtra}
%   \setmainfont[Mapping=tex-text, Ligatures={Common}, Numbers={OldStyle}]{Adobe Garamond Pro}
%   \setmonofont[Scale=0.72, Ligatures=NoCommon]{Monaco}
%   \setsansfont[Scale=0.9, Ligatures=NoCommon]{Gill Sans}
%   \defaultfontfeatures{Scale=MatchLowercase}

% \usepackage{unicode-math}
%   \setmathfont{STIXGeneral}

\usepackage[colorlinks, citecolor=citeBlue]{hyperref}
\usepackage{url}

\addtolength{\voffset}{-2cm}
\addtolength{\hoffset}{-1.5cm}
\addtolength{\textwidth}{3cm}
\addtolength{\textheight}{4cm}

\usepackage{color}
\definecolor{citeBlue}{rgb}{0,0,1}
% WIP colors
\definecolor{orange}{rgb}{1,0.5,0}
\definecolor{red}{rgb}{1,0,0}
\definecolor{blue}{rgb}{0,0,0.5}


\newcommand{\ail}{\textsf{Ail}}
\newcommand{\core}{\textsf{Core}}
\newcommand{\theC}{\textsf{C}}
\newcommand{\Csem}{\textsf{Csem}}

\newcommand{\IBMPOWER}{IBM POWER}
\newcommand{\POWER}{POWER}
\newcommand{\Coq}{Coq}
\newcommand{\Lem}{Lem}
\newcommand{\OCaml}{OCaml}
\newcommand{\HOL}{HOL4}
\newcommand{\Isabelle}{Isabelle}
\newcommand{\LLVM}{LLVM}
\newcommand{\GCC}{GCC}
\newcommand{\CCured}{CCured}

% \newcommand{\IBMPOWER}{\textsf{IBM POWER}}
% \newcommand{\POWER}{\textsf{POWER}}
% \newcommand{\Coq}{\textsf{Coq}}
% \newcommand{\Lem}{\textsf{Lem}}
% \newcommand{\OCaml}{\textsf{OCaml}}
% \newcommand{\HOL}{\textsf{HOL4}}
% \newcommand{\Isabelle}{\textsf{Isabelle}}
% \newcommand{\LLVM}{\textsf{LLVM}}
% \newcommand{\GCC}{\textsf{GCC}}
% \newcommand{\CCured}{\textsf{CCured}}


\newcommand{\TODO}{\textsc{ToDo}}


\title{Second Year Report \& Dissertation Schedule}
\author{Kayvan Memarian}
\date{August 2013}


\begin{document}

\maketitle

\section*{Context}

{\color{blue} The project laid out, in my first year report was the
  developpement of an executable formal semantics for the C
  programming language. While there exists past work on this topic,
  and several formalizations of C have been produced, the present work
  focuses on two aspects which have not really been dealt with in the
  past. First we consider that a formal model of the C language should
  not make any assumption regarding implementation defined choices
  (aspects of the language that the C standard leaves up to the
  implementor to define). Second, we would like a semantics capable of
  dealing with realistic low level programs, like pre-existing
  fragments of operating systems.}

To answer to this problem, I have opted for a {\it elaborative}
approach, where the semantics of C is given through a translation
function to a simpler and more principled language (called Core)
designed specifically for this purpose, and for which a direct
semantics is given.

Most of my work during my second year has been focused on fixing the
design for the Core language. Adding successively new features as they
where needed to make an accurate translation function from C
possible. 


\section*{Research activity}

\subsection*{Implementing Core and adding control operators}
During the first month following the submission of my first year
report, my focus was the implementation of Core's dynamics in the
\texttt{Lem} specification language. This was integrated on Csem,
a tool started by Justus Matthiesen and that I took over during my
first year.

The tool can be though of as a C compiler and evaluator: it takes as
input a preprocessed C program; translates it into a intermediate
language very close to the actual C; performs the C typechecking and
type annotate the program; then the translation to Core is then run
and finally the Core implementation evaluates the generated Core
program.

My original design of Core did not provide special constructs for the
elaboration of C's control features like loops, switches, gotos and in
particular the \texttt{break} and \texttt{continue} statements. The
original plan was to model these features purely through the use of
recursive Core functions. As a result the original implementation of
the Core's dynamics followed a traditional small-step style.  While
that approach kept the dynamics of Core simple (there was no need to
talk about continuations in it) and was nicely compositional, it had
the undesirable side-effect of requiring in the elaboration function
from C to Core to introduces significant duplications of code and a
loss of the original program structure through the creation of a large
quantity of auxiliary functions (e.g. for each \texttt{return}
statement in a C function, one would need to create a separate Core
function containing the elaboration of the prefix of the C
function up to that particular \texttt{return}). As an alternative we
added to Core a pair of constructs similar to labeled and goto
statements. Since memory objects and in particular their lifetime are
made precise in the syntax of Core, these constructs have slightly
convoluted semantics as it does include the creation and termination
of objects life. For example, in C one can have the following situation:
\begin{verbatim}
int *p;
{
  int x;
  p = &x;
  goto label;
}

...

label:
  S;
\end{verbatim}

Here when the execution enter the block (region surrounded by braces),
a new memory object is created corresponding to \texttt{x}, and the
following assignment makes to global pointer \texttt{p} to that memory
object. Then, the \texttt{goto} statement jumps out of the block,
ending at the same time the lifetime of local memory object. At this
pointer, the C standard says that the value of the pointer become
{\it indeterminate}. This is precisely something we aim accurately
modeling with our semantics. To do so, we therefore need to somehow
propagate throught the C to Core elaboration some informations
regarding block-scope C variable. This cannot be directly as Core does
not provide memory reference (with is C's ``variables'' really are),
instead we have symbolic names, some of them corresponding to the
memory location of objects (eg. in the elaboration of the beginning of
the lifetime of a C object, a specific C memory allocating action is
performed and its value is binded to a fresh symbolic names). As a
result, Core's labeling and goto statements are annotated with the
relevant symbolic names in way allowing us simulate lambda lifting.

A side effect of this new approach to modelling C's control operator,
is a loss of some compositionally at the level of C functions, since
the elaboration from C to Core now needs to pre-calculate the
continuations associated to labels (this is made unavoidable by
forward jumps) and the relation between block scoped object and Core
symbolic names.

Another related construct we added to Core is one analogous to C's
\texttt{return}. The addition of these three construsts radically
changed the shaped of Core's dynamics which is now implemented as a
small-step evaluator working on stack of continuations.


\subsection*{Indeterminately sequenced expressions}

In the first year report, I mentionned C function calls and the fact
that regarding their ordering the C standard says that they are ``{\it
  indeterminately-sequenced} with actions for which other operators do
not already enforce any ordering''. This combines with the semantics
of the post increment and decrement operators whose two actions (load
and incrementing/decrementing store) are defined to be atomic from the
point of view of an indeterminately-sequence function call. To model
this, Core has the special ordering operator ($\rhd$), for
increments/decrements, and the unary operator ($\{ \cdot \}$), marking
indeterminately-sequenced expression.

When we want to be actually run a Core expression with some
indeterminate sequencing, we need to enumerate all possible induced
sequencing graph. For example, in \texttt{x++~+~\{ E \}} we have two
possible sequencing graph: one where the expression E occurs before
the increment, and one where it occurs after.

One way to solve this problem is to take the graph semantics of the
Core expression (actions are nodes; the sequencing operator adds an
edge from the nodes of its first operand to the nodes of its second
operand; the unsequencing operator adds no edge; \dots), isolate
indeterminately sequenced actions and find all ways edges that can
added to the graph to make everything strictly ordered. The problem
here is again the lack of compositionality. Instead, I have wrote a
compositional Core to Core rewriting system, which in the pipeline of
Csem comes after a simple type-checker for Core but before the
evaluator and produce the set of possible Core programs (or introduce
non-determinism inside Core expressions).



\subsection*{Tool engineering}

I have also spent a fair amount of my time doing general software
engineering on Csem. We use to use are own C parser (other works tend
to make use of the CIL framework, but prefered not to rely on it as
its parser doesn't follow clearly what the standard describes) but we
did not support struct/union, typedef, compound expressions  and the
multitude of ways to write constant. Since making a C complete parser
is notoriously difficult, we opted to adopt a thrustworthy parser
produces in the context of the Compcert project.

I have also worked on adding to Csem the infrastructure that will
allow us to use it as a tool for detecting programs dependancy on
implementation defined behaviours. In particular by adding a Core
standard library and the mechanism to import implementation defined
choices definition files.


\subsection*{Memory layout model}

From the point of view of our semantic model of C, the dynamics of
Core described earlier allow us to specify what memory actions occurs
during the evaluation of C expressions and statements, and what
ordering constraints must be respected. This leaves out the details of
where and how the memory objects are stored, accessed and
modified. This aspect is dealt with by a memory layout model. This can
be made in component of our semantic tool (Csem) that remains
independent of the other components (like the C typechecking, or the
dynamics of C implemented by the elaboration function and Core's
implementation).

The development of such a memory model for Csem has been another focus
of my work. The main challenge comes from the fact that the C standard
tries to describe a model which on one side, the respect of some hight
level properties are impossed on implementation (like guaranties on
the absence 





{\color {red}
rhoncus tincidunt ligula, et iaculis enim consectetur ac. Etiam luctus
sem sem, vitae lacinia eros lobortis a.

mattis mollis, massa leo cursus ipsum, vitae malesuada mi ligula non
turpis. Duis pellentesque nibh purus, eu pharetra risus molestie
at. Nulla mauris eros, bibendum in ipsum et, dapibus fermentum
quam. Praesent ut vestibulum nisi, sed faucibus ante. Duis a tellus
leo.

Aenean quis scelerisque felis. Curabitur pulvinar elit id turpis
vestibulum, at pulvinar sem dignissim. Suspendisse leo leo,
ullamcorper eget urna eu, sollicitudin pharetra orci. Sed lacinia
lectus ligula, malesuada pharetra nulla mollis vitae. Morbi nulla
nisi, facilisis eget sem non, ultricies suscipit ipsum. Donec vitae
eleifend felis. Maecenas vestibulum sollicitudin massa, nec tristique
metus ultricies a. Nam scelerisque massa vel commodo elementum. Aenean
sed vestibulum arcu. Nulla mattis erat vitae rhoncus iaculis. Nullam
at laoreet risus, nec porttitor quam. Mauris ac consectetur
felis. Proin dolor enim, accumsan in condimentum ac, condimentum ut
lectus. Nunc pretium augue a condimentum sagittis. Morbi lacinia
sodales tristique.

Aliquam erat volutpat. Nulla facilisi. In sagittis lacus ornare mauris
rhoncus laoreet. Fusce ut neque accumsan, imperdiet nisi ut,
condimentum dui. Sed mattis, nulla ac venenatis venenatis, nulla eros
auctor lorem, id commodo ipsum turpis vel tortor. Morbi molestie
tincidunt hendrerit. Vivamus eu est varius, mollis justo eleifend,
mattis nisi. Cras tempus sodales purus eget pretium. Nulla imperdiet
in felis a facilisis. Donec congue est rhoncus, blandit enim non,
posuere lacus. Donec pharetra in nisi sed convallis.}

At the moment, our memory model only exists as a mix of prose and
non-operational Lem code. The current version of Csem relies on a
naive memory model (map from memory objects to values) which has
sufficient to evaluate ordering related part of the semantics and
simple expressions. In the short term, I am planning to polish and
translate to operational Lem code our real memory layout to be able to
connect it to the rest of Csem.


\subsection*{}

In the short term, I plan polish the Csem tool, a lots of hole remains
in the elaboration function, and add the memory model as just
described. Once this is done, I will be able to do large scale testing
of the semantics (so far I have only relied on small hand written
programs). This will probably make us of Regehr's Csmith tool.

In goal of producing out of Csem a tool for the detection of program
dependency on implementations, I plan to modify the Core evaluator by
adding a symbolic evaluator. This will allow to continue program
execution until implementation defined constant are really needed.

Another short term project is to connect Csem's to an operational
version of Mark Batty's concurency memory model that has be developped
last year by a master student.


\subsection*{Thesis layout}

{\color{red} Pellentesque non metus dictum, fringilla elit sed,
  pharetra magna. Aliquam erat volutpat. Aliquam porttitor arcu justo,
  a molestie mauris tempor sit amet. Phasellus non laoreet dui,
  pellentesque faucibus ante. Maecenas id erat at est bibendum
  rhoncus. Quisque lorem elit, commodo quis porta non, in- terdum ac
  nisl. Ut tempor commodo nibh, sed euismod nunc adipiscing
  sed. Aliquam pulvinar, nulla vel mattis mollis, massa leo cursus
  ipsum, vitae malesuada mi ligula non turpis. Duis pellentesque nibh
  purus, eu pharetra risus molestie at. Nulla mauris eros, bibendum in
  ipsum et, dapibus fermentum quam. Praesent ut vestibulum nisi, sed
  faucibus ante. Duis a tellus leo.  Aenean quis scelerisque
  felis. Curabitur pulvinar elit id turpis vestibulum, at pulvinar sem
  dignissim. Suspendisse leo leo, ullamcorper eget urna eu,
  sollicitudin pharetra orci. Sed la- cinia lectus ligula, malesuada
  pharetra nulla mollis vitae. Morbi nulla nisi, facilisis eget sem
  non, ultricies suscipit ipsum. Donec vitae eleifend felis. Maecenas
  vestibulum sollicitudin massa, nec tristique metus ultricies a. Nam
  scelerisque massa vel com- modo elementum. Aenean sed vestibulum
  arcu. Nulla mattis erat vitae rhoncus iaculis. Nullam at laoreet
  risus, nec porttitor quam. Mauris ac consectetur felis. Proin dolor
  enim, accumsan in condimentum ac, condimentum ut lectus. Nunc
  pretium augue a condimentum sagittis. Morbi lacinia sodales
  tristique.  Aliquam erat volutpat. Nulla facilisi. In sagittis lacus
  ornare mauris rhoncus laoreet. Fusce ut neque accumsan, imperdiet
  nisi ut, condimentum dui. Sed mattis, nulla ac venenatis ven-
  enatis, nulla eros auctor lorem, id commodo ipsum turpis vel
  tortor. Morbi molestie tincidunt hendrerit. Vivamus eu est varius,
  mollis justo eleifend, mattis nisi. Cras tempus sodales purus eget
  pretium. Nulla imperdiet in felis a facilisis. Donec congue est
  rhoncus, blandit enim non, posuere lacus. Donec pharetra in nisi sed
  convallis.  Aenean quis scelerisque felis. Curabitur pulvinar elit
  id turpis vestibulum, at pulvinar sem dignissim. Suspendisse leo
  leo, ullamcorper eget urna eu, sollicitudin pharetra orci. Sed la-
  cinia lectus ligula, malesuada pharetra nulla mollis vitae. Morbi
  nulla nisi, facilisis eget sem non, ultricies suscipit ipsum. Donec
  vitae eleifend felis. Maecenas vestibulum sollicitudin massa, nec
  tristique metus ultricies a. Nam scelerisque massa vel com- modo
  elementum. Aenean sed vestibulum arcu. Nulla mattis erat vitae
  rhoncus iaculis. Nullam at laoreet risus, nec porttitor quam. Morbi
  nulla nisi, facilisis eget sem non, ultricies suscipit ipsum.}






























\end{document}
